\documentclass{ltjsarticle}
\usepackage{graphicx} % Required for inserting images
\usepackage{amsmath} % 数式関係のパッケージ

% % はコメントです. コンパイルの際には無視されます.

\title{演習 I 第 4 回}
\author{岡田 真}
\date{\today}

\begin{document}
\maketitle

\section{Introduction}
\verb|\section{}| で章を始めます. 
改行では改行されません. 

空白で改行されます. 

\section{数式}

文書中に数式 $a=b+c$ を書きたい場合は,\$で数式を挟みます. 

行替えを行って数式を書く場合は,\verb|\begin{equation}, \end{equation}|を使います.

式\ref{eq:abc}は式の例です.

\begin{equation} \label{eq:abc}
a=b+c 
\end{equation} 

式\ref{eq:abcsep}は複数行の式の例です.

\begin{equation} \label{eq:abcsep}
\begin{split}
a &= b+c\\
  &= d+e
\end{split}
\end{equation} 

色々な数式表現: $x^2$, $x^{10}$, $A_n$, $\frac{1+x}{2-y}$

\section{表の書き方}

表は table と tabular で書きます. 

表\ref{tab:baseball}は野球の順位の表です. 

\begin{table}[tb] 
\caption{プロ野球セントラルリーグ2016年9月29日の順位}
\label{tab:baseball}
\centering 
\begin{tabular}{|l|l|r|r|} 
\hline 
順位&チーム名&勝率&差\\ 
\hline\hline 
1&広島東洋&.626&-\\ 
\hline 
2&読売&.518&15.0\\ 
\hline 
3&横浜DeNA&.496&18.0\\ 
\hline 
4&東京ヤクルト&.450&24.5\\ 
\hline 
5&阪神&.449&24.5\\ 
\hline 
6&中日&.414&29.5\\ 
\hline 
\end{tabular} 
\end{table} 

\section{図の貼り方} 

図はincludegraphicsを使って貼り付けします. 

{\TeX} で {\TeX} のロゴが出力されます. 
{\TeX} で用いられる画像ファイル形式は主にPDFです. 

図\ref{fig:apple} はリンゴです.

\begin{figure}[tb] 
    \centering 
    \includegraphics[width =0.8\hsize]{./apple.pdf} 
    \caption{リンゴ} 
    \label{fig:apple} 
\end{figure}

\end{document}
